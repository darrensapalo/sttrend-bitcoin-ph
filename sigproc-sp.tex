% THIS IS SIGPROC-SP.TEX - VERSION 3.1
% WORKS WITH V3.2SP OF ACM_PROC_ARTICLE-SP.CLS
% APRIL 2009
%
% It is an example file showing how to use the 'acm_proc_article-sp.cls' V3.2SP
% LaTeX2e document class file for Conference Proceedings submissions.
% ----------------------------------------------------------------------------------------------------------------
% This .tex file (and associated .cls V3.2SP) *DOES NOT* produce:
%       1) The Permission Statement
%       2) The Conference (location) Info information
%       3) The Copyright Line with ACM data
%       4) Page numbering
% ---------------------------------------------------------------------------------------------------------------
% It is an example which *does* use the .bib file (from which the .bbl file
% is produced).
% REMEMBER HOWEVER: After having produced the .bbl file,
% and prior to final submission,
% you need to 'insert'  your .bbl file into your source .tex file so as to provide
% ONE 'self-contained' source file.
%
% Questions regarding SIGS should be sent to
% Adrienne Griscti ---> griscti@acm.org
%
% Questions/suggestions regarding the guidelines, .tex and .cls files, etc. to
% Gerald Murray ---> murray@hq.acm.org
%
% For tracking purposes - this is V3.1SP - APRIL 2009

\documentclass{acm_proc_article-sp}

\usepackage{graphicx}

\begin{document}

\title{Increasing Bitcoin Adoption In The Philippines}

\numberofauthors{3} 
\author{
\alignauthor
Roberto Basil E. Cruz\\
       \affaddr{De La Salle University}\\
       \email{roberto\_{}cruz@dlsu.edu.ph}
% 2nd. author
\alignauthor
\alignauthor Dominic William B. Elayda\\
       \affaddr{De La Salle University}\\
       \email{dominic\_{}elaydaiii@dlsu.edu.ph}
% 3rd. author
\and
\alignauthor
Darren Karl A. Sapalo\\
       \affaddr{De La Salle University}\\
       \email{darren\_{}sapalo@dlsu.edu.ph}
  % use '\and' if you need 'another row' of author names
% 4th. author
}
% There's nothing stopping you putting the seventh, eighth, etc.
% author on the opening page (as the 'third row') but we ask,
% for aesthetic reasons that you place these 'additional authors'
% in the \additional authors block, viz.
\date{8 December 2014}
% Just remember to make sure that the TOTAL number of authors
% is the number that will appear on the first page PLUS the
% number that will appear in the \additionalauthors section.

\maketitle
\begin{abstract}
As a country whose economic growth greatly depends on its OFWs, the Philippines can greatly benefit from the adoption of Bitcoin as a new means for fund transferring and online purchasing. This paper begins by discusses the Bitcoin and other cryptocurrencies, their advantages and disadvantages. It follows with various payment systems in the Philippine setting, and the proposed approach to improve Bitcoin adoption. The paper analyzes the cost estimates and possible risks that may be encountered by the proposed approach, and ends with a summary of the proposed approach.

\end{abstract}


\keywords{Bitcoin, Cryptocurrency, Money, Fund transferring, Online purchasing}
% NOT required for Proceedings

\section{Motivation}

The Philippine Economy has been steadily increasing for the past decade. From 2001 to 2014, the Gross Domestic Product (GDP) of the Philippines increased annually by an average of 5.02\%, with an all-time high of 8.90\% in 2010 \cite{TradingEconomics:11272014}. According to some analysts, the Philippines is set to become one of the largest economies in Southeast Asia by 2050 \cite{HSBC:01112012}. This trend in economic growth has been shown to have a negative relationship with unemployment. Low unemployment typically lead to higher value of goods and services and thus resulting in a healthier economy. This relationship often results in higher income for the nation and thus more spending power for consumers \cite{EconomicGrowth:01072013}. With potentially increasing expenditures, we believe there needs to be more options for consumers to spend their money. This means opening up more markets and payment methods for the Filipino people. One of these payment methods is the digital cryptocurrency known as Bitcoin.

In this paper, we present a possible solution on increasing the adoption rate of Bitcoin in the Philippines. Section 2 discusses the idea of Bitcoin in general and how it compares to the different payment methods that currently exist in the market today. We will also be looking at the benefits and limitations of each. In section 3, we present our proposed solution on how to increase the adoption of Bitcoin in the Philippines today. We also go over an in-depth analysis of this and discuss the feasibility of the implementation of this solution in Section 4. Finally, Section 5 contains our concluding remarks and recommendations.


\section{Background}

In this section, we give an overview of the idea behind cryptocurrencies. We also look into existing implementations of cryptocurrencies such as Bitcoin, Ripple and Litecoin and how they differ from each other. We will also look into existing mobile payment systems currently in place today as an alternative means of payment and how they compare to using cryptocurrencies.

\subsection{Cryptocurrencies}
Cryptocurrency is an alternative medium used in exchange of goods or services. It is a form of currency that is only stored, created and managed electronically, known as digital currency. Like the traditional medium of exchange such as money, the purpose of cryptocurrency is to be used to buy physical goods and services. It uses cryptography to create new units of the currency and to secure the transactions that take place, hence the name.

The idea of a cryptocurrency was conceptualized as early as 1998 by Wei Dai on the cypherpunks mailing list. He proposed a new form of money that relies on cryptography to control its creation and transaction, rather than having it managed by a centralized authority. The first implementation of cryptocurrency was created in 2009, known as Bitcoin \cite{Bitcoin:2014}.

One feature typically found in cryptocurrencies is its decentralized nature. Performing exchange is done by recording transactions in a public ledger. These transaction operate on a peer-to-peer network, with no centralized regulating body. This allows for pseudo-anonymous exchanges between parties.

While cryptocurrencies arrive in an electronic form, they are not the same as electronic money. The main difference is that electronic money still the same medium as the physical money, only represented in a digital form for ease of transferring. The creation of this currency is still regulated and done by a centralized, governing body - something not appliable in most cryptocurrencies.


\subsubsection{Bitcoin}

Bitcoin was launched in 2009 by Satoshi Nakomoto as ``an alternative to fiat currencies'' \cite{Worldbank:2014}. Bitcoins are earned through a process called mining, in which computers use their processing power to solve complex mathematical calculations and to keep track of the many transactions added to the ledger to receive new Bitcoins. The miners compete with each other to find the solution of the complex calculations.  When a solution of the block of transaction is found (usually transactions every 10 minutes), the system that solved it is rewarded a block or 25 Bitcoins. This number of newly made Bitcoins issued is halved every four years, so by the year 2016, the amount of new Bitcoins made would be 12.5. This will be the case until there are a total of 21 million Bitcoins in the world.

To obtain bitcoins from non-mining, the most common way is to exchange fiat currencies to bitcoin and people can store it into a Bitcoin wallet, which can be desktop, mobile or web-based. This medium of purchasing bitcoins, however, can be susceptible to hackers or scammers. Using these wallets, buying items online is simply just transferring bitcoins from one wallet to another \cite{Coindesk:03062014b}. Transactions made with bitcoins are safe since the public ledger (called the block chain) can be seen by all and cannot be tampered with. In order to send a Bitcoin to another wallet, two things are needed: the Bitcoin address and and a private key that are generated randomly. For example, if Alice wants to send Bob Bitcoins, she needs to use her private key to send a message to Bob's Bitcoin address that she will give him this amount of bitcoins. This transferal would not happen instantaneously as the miners need to validate the transaction made. This could take about 10 minutes or more (since mining blocks occurs about every 10 minutes).

The use of Bitcoin offers a number of advantages over using traditional mediums of exchange \cite{Bitcoin:2014}:

\begin{enumerate}
	\item \textbf{Payment freedom} It is possible to send and receive any amount of money instantly anywhere in the world at any time.
	\item \textbf{Very low fees} Bitcoin payments are currently processed with either no fees or extremely small fees. Users may include fees with transactions to receive priority processing, which results in faster confirmation of transactions by the network. Additionally, merchant processors exist to assist merchants in processing transactions, converting bitcoins to fiat currency and depositing funds directly into merchants' bank accounts daily. As these services are based on Bitcoin, they can be offered for much lower fees than with PayPal or credit card networks.
	\item \textbf{Fewer risks for merchants} Bitcoin transactions are secure, irreversible, and do not contain customers’ sensitive or personal information. This protects merchants from losses caused by fraud or fraudulent chargebacks, and there is no need for PCI compliance. Merchants can easily expand to new markets where either credit cards are not available or fraud rates are unacceptably high. The net results are lower fees, larger markets, and fewer administrative costs.
	\item \textbf{Security and control} Bitcoin users are in full control of their transactions; it is impossible for merchants to force unwanted or unnoticed charges as can happen with other payment methods. Bitcoin payments can be made without personal information tied to the transaction. This offers strong protection against identity theft. Bitcoin users can also protect their money with backup and encryption.
	\item \textbf{Transparent and neutral} All information concerning the Bitcoin money supply itself is readily available on the block chain for anybody to verify and use in real-time. No individual or organization can control or manipulate the Bitcoin protocol because it is cryptographically secure. This allows the core of Bitcoin to be trusted for being completely neutral, transparent and predictable.
\end{enumerate}


Although the use of Bitcoin is advantageous in various aspects, there are also disadvantages in using it \cite{Bitcoin:2014}:
\begin{enumerate}
	\item \textbf{Degree of acceptance} Many people are still unaware of Bitcoin. Every day, more businesses accept bitcoins because they want the advantages of doing so, but the list remains small and still needs to grow in order to benefit from network effects.
	\item \textbf{Volatility} The total value of bitcoins in circulation and the number of businesses using Bitcoin are still very small compared to what they could be. Therefore, relatively small events, trades, or business activities can significantly affect the price. In theory, this volatility will decrease as Bitcoin markets and the technology matures.
	\item \textbf{Ongoing development} Bitcoin software is still in beta with many incomplete features in active development. New tools, features, and services are being developed to make Bitcoin more secure and accessible to the masses. Some of these are still not ready for everyone. Most Bitcoin businesses are new and still offer no insurance. In general, Bitcoin is still in the process of maturing.
\end{enumerate}


\subsubsection{Other cryptocurrencies}
Since the advent of Bitcoin in 2009, other cryptocurrencies followed suit. Numerous cryptocurrency specifications have been published, and a significant number of them are derived from the first implementation of Bitcoin. The two most widely adopted cryptocurrencies next to Bitcoin are Ripple and Litecoin. Table~\ref{cryptocurrencies} shows the top 3 most widely adopted cryptocurrencies as of November 2014. Note that Market Cap and Price per Unit is in USD.


\begin{table}[h]
\centering
\begin{tabular}{ | l | l | l | l | }
\hline
  Name & Market Cap & Price / Unit & Available Supply \\ \hline 
  Bitcoin   & 4,970,073,323 & 365.84 & 13.5M BTC \\ \hline
  Ripple    & 415,005,339 & 0.013439 & 30.8B XRP \\ \hline
  Litecoin  & 126,835,494 & 3.67 & 34.5M LTC \\ \hline
\end{tabular}
  \caption{The top 3 most widely adopted cryptocurrencies as of November 2014. }
  \label{cryptocurrencies}
\end{table}

At its core, majority of these different cryptocurrency implementations all operate in a similar fashion (technically speaking). Also, these implementations of cryptocurrencies are still at a very early stage, and are likely subject to change in the future in order to adapt to the volatile state of the market. Right now, what makes a certain cryptocurrency stand out from the rest simply boils down to one thing - adoption.

\cite{Coins.ph:06292014} discusses the bitcoin adoption in the Philippines as it is led by a well-known startup: Coins.ph. Ron Hose, co-founder and CEO of Coins.ph, firmly believes that bitcoin can help a great range of consumers in the Philippines. He explains that ``credit card penetration in the Philippines is 3\%, which means only three out of a hundred customers that land on an e-commerce site have an immediate way to pay for it'' \cite{Coins.ph:06292014}. A larger pain point of Filipinos would be transferring money from other countries. An estimate of around \$21.3 billion was sent back to the country in 2010, a large portion of which are going to people who do not have bank accounts. These people have to collect money in person at retail locations like Western Union which requires a payment of 9\% in fees. Hose explains that ``the reason they pay this 9\% is because they don't have a bank account and likely will never have a bank account because they do not have enough savings.'' He further explains that that 9\% is equivalent in three-to-four days of income as payment just to pick up your money.

Coins.ph is a bitcoin exchange service and merchant processor which now services two major Filipino Merchants: MetroDeal and CashCashPinoy \cite{Coins.ph:03202014}. With the announcement of the new payment method accepted by MetroDeal, the Philippine's top daily deals site, Bangko Sentral ng Pilipinas (BSP) warns its consumers about the use of digital currencies such as its high volatility and lack of protection for its consumers. Furthermore, this emerging technology has yet to have regulations.

\subsection{Related Payment Systems}
The following section discusses various modes of payment available in the Philippines. Their advantages and disadvantages are discussed, along with their respective transaction fees.

\subsubsection{Cash and Credit cards}
The standard mode of payment which all commercial stores accept is cash. This is the preferred and often cheaper mode of payment because cash is simply money already. Credit cards gives the users the convenience of bringing a card instead of a large amount of money, providing a sense of security for payment. To fund this service, credit card companies or banks require membership fees, interest charges, and sometimes charge with varying foreign conversion rates. Most credit cards partner with leading commercial stores to give a rewards or incentives program to encourage users to use their service.

\subsubsection{Smart Money}
Smart Money is a electronic wallet for Smart or Talk N' Text subscribers \cite{Smart:02092014}. In order to send or receive money, the user has to link a bank account to his mobile number. This system can also allow the user to shop online, withdraw from ATMs and pay bills. A limitation for this medium is that transferring money is only limited to Smart and Talk N' Text subscribers and this medium has a transfer fee of about 0.5\% of the money transferred. A workaround for the limited transferring at present is that vendors transfer Smart Money to other vendors for real money plus a transfer fee more than 0.5\% in order to profit from the service.

\subsubsection{GCash}
GCash is similar to Smart Money but is limited to Globe and Touch Mobile subscribers. It allows users to ``send money, pay bills, buy load, enjoy rebates, shop online, play games, and donate to your favorite institution from your mobile phone,'' \cite{Globe:02082014}. It has an advantage over other modes of payment in that it allows users to register for the service without attaching a credit card or a bank account, allowing the service to be highly accessible to many Filipinos. The transaction costs are free for uses such as paying bills, paying BIR, buying load, and others. However, a fee of 1\%, or PHP 10.00 for every PHP 1000.00, to perform cash transfers to another GCash user, or 2\%, PHP 20.00 for every PHP 1000.00, when you send money to an outlet or cash out. 

\subsubsection{Coins.ph}
Coins.ph is a bitcoin exchange service and merchant processor \cite{Coins.ph:06292014}. It allows users to register for a bitcoin wallet for free. Coins.ph provides various methods of buying bitcoins \cite{Coins.ph:12042014}. Coin such as walk-in cash deposits to various banks (BDO, BPI, Security Bank, Union Bank), onlien transfers (BDO, BPI), deposits via ATM  machines (BDO, BPI), cash payments (M Lhuillier, Cebuana Lhuillier, SM Bills payment centers, bayad centers, and LBC outlets), or purchasing via GCash. Alternatively, you can turn your bitcoins back to cash by selling bitcoins back to Coins.ph and can receive the payment through various ways such as bank deposit, cash puck up (LBC, M Lhuillier, Cebuana Lhuillier), or converting it to GCash. Note that you can receive your cash even without a bank account.

Coins.ph conveniently lists down partners who accept Bitcoin as payment for various producst and services such as food, travel, electronics, web services and more \cite{Coins.ph:12092014}. Aside from purchasing products, Coins.ph also allows you to send cash to friends.

Each of the payment systems presented in this section normally fall under one of three categories - banking, mobile payments, and online transactions. The existing payment methods used in the Philippines cover banking (with traditional cash and credit cards) and mobile payments (with G-Cash and Smart Money) very well. Online transactions, however, are still hindered by a number of road blocks. In order to take part in a majority of online transactions, there are two key components necessary - internet access and a banking facility that supports it.

Internet access is highly instrumental in facilitating online payments. The availability of an internet connection in the Philippines has slowly been increasing. However, it still remains quite low, with only 37\% of the nation having regular access to the internet as of 2013 \cite{Techinasia:01092014}. The banking and credit card penetration rates are even lower. Only 27\% of Filipinos have an account at a formal financial institution, and 3\% have access to a credit card as of 2012. With these figures, the markets available to Filipino buyers are significantly limited by those they can obtain locally \cite{ABSCBN:09232014}. By introducing Bitcoin as an alternative means of payment, this can be alleviated. With the country having the one of the lowest credit card penetration rates in Asia Pacific, this opens up huge potential for other payment schemes. However, one of the main problems of Bitcoin here in the Philippines is that it is not widely adopted yet. As a result, consumers and merchants do not have much incentive in using the currency.

In the next section, we propose ways on how to increase the adoption of Bitcoin in the Philippines in order to expand the markets available to Filipino buyers.

\section{Proposed Solution}

An ideal Bitcoin scene in the Philippines is to be highly accessible for the large masses of Filipinos: enabling users to buy and sell Bitcoins as easy as it is to buy and sell mobile load in nearby sari-sari stores or convenience stores such as 7-Eleven, Mini-Stop, or Family Mart. This highly convenient way of transferring funds and purchasing products will open the pathways for a larger number of Filipino citizens to purchase things online thus improving the market of e-commerce web portals. Furthermore, fund transfers that are this easily accessible will enable relatives of OFWs to receive remittances within a day and with less fees.

To understand what users currently know about Bitcoin, a survey within certain selected cities such as Makati, Ortigas, Pasay, Paranaque, and Taguig can be conducted to study what digital payment methods Filipinos are aware of and have used before. Personal information will be collected to profile the respondent so that a balanced sample can sufficiently represent different kinds of online Filipino Netizens.  The same survey can be conducted with the same respondents to gather their initial impressions and feedback on the usage of the technology, so as to guide Bitcoin and online merchants with their development. One of the biggest challenges is to be able to get the support of various merchants such as restaurants, retail shops, or entertainment centers that one finds in highly developed commercial areas. With the leading developers

Once we have identified what users do not know about Bitcoin, we can now address this with proper education and information dissemination. The problems of most emerging technologies is that many users are afraid of shifting into new and unfamiliar things because of fear of what is unknown. This is why our first step is to find out what it is users do not understand and to address this lack of information. Once we know what level of information most users know about Bitcoin, the next step is to educate the users through a medium such as a web portal, providing the option for either a concise or a comprehensive definition or description of what Bitcoin is and why it is a viable and effective technology. Allowing the users to have either of the two is an important step because we want to be able to cater to non-technical users who simply need to know the basics of Bitcoin. This education can be facilitated through various media such as promoting infographics on social media, hosting promotional events in different kinds of media such as social media or on the radio. 

Enabling the citizens to learn more about Bitcoin as an option for fund transferring or purchasing online is especially important since the Philippines supports a large number of OFWs.  Primarily, we want to be able to target the large market of OFWs who send back money to their relatives. To encourage Filipinos to use Bitcoin, they must be informed of the key advantages of this new technology over traditional fund transferring methods. For example, a clear advantage with this technology is bringing home almost a tenth more of your remittances compared to receiving money in other ways, as explained in section 2.1.2. 

The private sector does not have to be the sole proponents or advocates of Bitcoin. The merchants and the users are not the only ones who can benefit from Bitcoin, but the government as well. When the government works hand-in-hand with the private sector in promoting Bitcoin, the adoption rate of Bitcoin in the Philippines will improve much more. Upon learning that MetroDeal has accepted Bitcoins as payment \cite{Coins.ph:06292014}, the Bangko Sentral ng Pilipinas (BSP) warns its users about risks and concerns regarding the emerging technology. Aside from informing its users of the risks of this kind of technology, the government can aid Bitcoin adoption rate by providing laws and regulations to protect the citizens, the merchants, and the state. This can be done by studying how other countries and their governments have handled regulations for Bitcoin.

\section{Analysis}
In this section, the cost factors, potential risks, and sustainability issues are presented.

\subsection{Cost Estimates}
There are various costs for the events or activities to promote Bitcoin. The following lists down some of the possible activities that have either a recurring or a one time cost:

\begin{enumerate}
	\item \textbf{User research} This is essential in understanding our users: what they currently know and need.
	\item \textbf{Bitcoin research and presentation} For Bitcoin to be adopted by merchants such as restaurants, shops, and more, it is essential to be able to produce a short presentation as to the benefits and strategic advantages of the technology.
	\item \textbf{Graphic design} Short snippets of information such as well-designed infographics are extremely effective in educating users, whether over internet or print media.
	\item \textbf{Prizes for promotional events} Free items from sponsors and actual Bitcoin will encouraged users to take the first step forward into trying out Bitcoin.
	\item \textbf{Broadcasting the promotions in media} We want to reach a wide range of Filipinos. Having broadcasts for the promotional events on different kinds of media (such as radio or social media) will reach a greater audience.
\end{enumerate}

A brief estimate of the abovementioned costs is found on Table~\ref{cost_table}. Note that the prices are in Philippine Peso (PHP). 

The fees for user research includes the wages of the user researcher and incentives for participants of user research activities such as surveys. Bitcoin research includes the wages for the researcher or business analyst that will prepare a presentation to encourage merchants to adopt Bitcoin as a means of payment. Graphic design includes fees for producing infographics related to Bitcoin. Promotional events will include fees to fund the prizes, but prizes can include sponsor items. Broadcasting in media is for the fees for advertising on radio media, or paying for a social media manager to post Bitcoin related posts every day.

\begin{table}[h]
\centering
\begin{tabular}{ | l | l | }
\hline
  Item & Cost estimate  \\ \hline \hline
  User research           & 30,000 PhP \\ \hline
  Bitcoin research        & 25,000 PhP \\ \hline
  Graphic design          & 20,000 PhP \\ \hline
  Promotional events      & 35,000 PhP \\ \hline
  Broadcasting in media   & 110,000 PhP \\ \hline
\end{tabular}
  \caption{The cost estimates for increasing Bitcoin adoption. }
  \label{cost_table}
\end{table}


\subsection{Risk Factors}
Bitcoin is approximately ``8 times as volatile as the stock market and close to 20 times more than the US dollar'' \cite{BitcoinMyths:08182014}. This may discourage potential users from adopting Bitcoin as an alternative to their electronic wallets. Moreover, Bitcoin is too small to affect the global economy. The total value of the bitcoins that exist today is about \$7 billion and the daily transaction volume is approximately \$50 million whereas the daily transaction volume of credit cards in 2012 was \$7.4 trillion \cite{BitcoinMyths:08182014}. Another risk is that people can abuse the anonymity of Bitcoin transactions and use it as a front for illicit activities such as soliciting drugs \cite{Worldbank:2014}. Furthermore, Bitcoin is at its infancy, and when it develops, it may have solved the aforementioned risks.

% a problem encountered would be that sari-sari stores or convenience stores would require Wi-Fi and internet capabilities to be able to perform these kinds of transactions.

\section{Conclusion}
The Philippine economy is steadily growing and more markets are opening up for consumers. However, their options for payment are limited due to the limited banking and credit card penetration rates of the country. There are existing mobile payment systems currently in place, however, these require a certain fee to use. Bitcoin is an open and decentralized cryptocurrency exchange system that can be seen as a viable alternative of payment. More merchants are starting to recognize Bitcoin and is increasingly becoming relevant as time goes by.

We proposed a solution on how to increase bitcoin adoption in the Philippines. The fundamental idea is to first make people understand the basic concept of Bitcoin and how it compares to other payment methods. A key factor in making this succeed is to make sure that Bitcoin is easily accessible to the masses. With the success of electronic loading or e-load merchants for mobile phone credits, we plan to piggybank on that infrastructure by having Bitcoin vendors allow franchising of their service through easily accessible establishments such as convenience stores and sari-sari stores.


\bibliographystyle{plain}
\bibliography{myreferences}  

\end{document}
