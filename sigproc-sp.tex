% THIS IS SIGPROC-SP.TEX - VERSION 3.1
% WORKS WITH V3.2SP OF ACM_PROC_ARTICLE-SP.CLS
% APRIL 2009
%
% It is an example file showing how to use the 'acm_proc_article-sp.cls' V3.2SP
% LaTeX2e document class file for Conference Proceedings submissions.
% ----------------------------------------------------------------------------------------------------------------
% This .tex file (and associated .cls V3.2SP) *DOES NOT* produce:
%       1) The Permission Statement
%       2) The Conference (location) Info information
%       3) The Copyright Line with ACM data
%       4) Page numbering
% ---------------------------------------------------------------------------------------------------------------
% It is an example which *does* use the .bib file (from which the .bbl file
% is produced).
% REMEMBER HOWEVER: After having produced the .bbl file,
% and prior to final submission,
% you need to 'insert'  your .bbl file into your source .tex file so as to provide
% ONE 'self-contained' source file.
%
% Questions regarding SIGS should be sent to
% Adrienne Griscti ---> griscti@acm.org
%
% Questions/suggestions regarding the guidelines, .tex and .cls files, etc. to
% Gerald Murray ---> murray@hq.acm.org
%
% For tracking purposes - this is V3.1SP - APRIL 2009

\documentclass{acm_proc_article-sp}

\usepackage{graphicx}

\begin{document}

\title{Increasing Bitcoin Adoption In The Philippines}

\numberofauthors{3} 
\author{
\alignauthor
Roberto Basil E. Cruz\\
       \affaddr{De La Salle University}\\
       \email{roberto\_{}cruz@dlsu.edu.ph}
% 2nd. author
\alignauthor
\alignauthor Dominic William B. Elayda\\
       \affaddr{De La Salle University}\\
       \email{dominic\_{}elaydaiii@dlsu.edu.ph}
% 3rd. author
\and
\alignauthor
Darren Karl A. Sapalo\\
       \affaddr{De La Salle University}\\
       \email{darren\_{}sapalo@dlsu.edu.ph}
  % use '\and' if you need 'another row' of author names
% 4th. author
}
% There's nothing stopping you putting the seventh, eighth, etc.
% author on the opening page (as the 'third row') but we ask,
% for aesthetic reasons that you place these 'additional authors'
% in the \additional authors block, viz.
\date{8 December 2014}
% Just remember to make sure that the TOTAL number of authors
% is the number that will appear on the first page PLUS the
% number that will appear in the \additionalauthors section.

\maketitle
\begin{abstract}

Insert abstract here.

\end{abstract}


\keywords{Bit coin, Cryptocurrency, money} 
% NOT required for Proceedings

\section{Motivation}

The Philippine Economy has been steadily increasing for the past decade. From 2001 to 2014, the Gross Domestic Product (GDP) of the Philippines increased annually by an average of 5.02\%, with an all-time high of 8.90\% in 2010 \cite{TradingEconomics:11272014}. According to some analysts, the Philippines is set to become one of the largest economies in Southeast Asia by 2050 \cite{HSBC:01112012}. This trend in economic growth has been shown to have a negative relationship with unemployment. Low unemployment typically lead to higher value of goods and services and thus resulting in a healthier economy. This relationship often results in higher income for the nation and thus more spending power for consumers \cite{EconomicGrowth:01072013}. With potentially increasing expenditures, we believe there needs to be more options for consumers to spend their money. This means opening up more markets and payment methods for the Filipino people. One of these payment methods is the digital cryptocurrency known as Bitcoin.

In this paper, we present a possible solution on increasing the adoption rate of Bitcoin in the Philippines. Section 2 discusses the idea of Bitcoin in general and how it compares to the different payment methods that currently exist in the market today. We will also be looking at the benefits and limitations of each. In section 3, we present our proposed solution on how to increase the adoption of Bitcoin in the Philippines today. We also go over an in-depth analysis of this and discuss the feasibility of the implementation of this solution in Section 4. Finally, Section 5 contains our concluding remarks and recommendations.


\section{Background}

In this section, we give an overview of the idea behind cryptocurrencies. We also look into existing implementations of cryptocurrencies such as Bitcoin, Ripple and Litecoin and how they differ from each other. We will also look into existing mobile payment systems currently in place today as an alternative means of payment and how they compare to using cryptocurrencies.

\subsection{Cryptocurrencies}
Cryptocurrency is an alternative medium used in exchange of goods or services. It is a form of currency that is only stored, created and managed electronically, known as digital currency. Like the traditional medium of exchange such as money, the purpose of cryptocurrency is to be used to buy physical goods and services. It uses cryptography to create new units of the currency and to secure the transactions that take place, hence the name.

The idea of a cryptocurrency was conceptualized as early as 1998 by Wei Dai on the cypherpunks mailing list. He proposed a new form of money that relies on cryptography to control its creation and transaction, rather than having it managed by a centralized authority. The first implementation of cryptocurrency was created in 2009, known as Bitcoin \cite{Bitcoin:2014}.

One feature typically found in cryptocurrencies is its decentralized nature. Performing exchange is done by recording transactions in a public ledger. These transaction operate on a peer-to-peer network, with no centralized regulating body. This allows for pseudo-anonymous exchanges between parties.

While cryptocurrencies arrive in an electronic form, they are not the same as electronic money. The main difference is that electronic money still the same medium as the physical money, only represented in a digital form for ease of transferring. The creation of this currency is still regulated and done by a centralized, governing body - something not appliable in most cryptocurrencies.


\subsubsection{Bitcoin}

Bitcoin was launched in 2009 by Satoshi Nakomoto as ``an alternative to fiat currencies'' \cite{Worldbank:2014}. Bitcoins are earned through a process called mining, in which computers use their processing power to solve complex mathematical calculations and to keep track of the many transactions added to the ledger to receive new Bitcoins. The miners compete with each other to find the solution of the complex calculations.  When a solution of the block of transaction is found (usually transactions every 10 minutes), the system that solved it is rewarded a block or 25 Bitcoins. This number of newly made Bitcoins issued is halved every four years, so by the year 2016, the amount of new Bitcoins made would be 12.5. This will be the case until there are a total of 21 million Bitcoins in the world.

To obtain bitcoins from non-mining, the most common way is to exchange fiat currencies to bitcoin and people can store it into a Bitcoin wallet, which can be desktop, mobile or web-based. This medium of purchasing bitcoins, however, can be susceptible to hackers or scammers. Using these wallets, buying items online is simply just transferring bitcoins from one wallet to another \cite{Coindesk:03062014b}. Transactions made with bitcoins are safe since the public ledger (called the block chain) can be seen by all and cannot be tampered with. In order to send a Bitcoin to another wallet, two things are needed: the Bitcoin address and and a private key that are generated randomly. For example, if Alice wants to send Bob Bitcoins, she needs to use her private key to send a message to Bob's Bitcoin address that she will give him this amount of bitcoins. This transferal would not happen instantaneously as the miners need to validate the transaction made. This could take about 10 minutes or more (since mining blocks occurs about every 10 minutes).

The use of Bitcoin offers a number of advantages over using traditional mediums of exchange \cite{Bitcoin:2014}:

\begin{enumerate}
	\item \textbf{Payment freedom} It is possible to send and receive any amount of money instantly anywhere in the world at any time.
	\item \textbf{Very low fees} Bitcoin payments are currently processed with either no fees or extremely small fees. Users may include fees with transactions to receive priority processing, which results in faster confirmation of transactions by the network. Additionally, merchant processors exist to assist merchants in processing transactions, converting bitcoins to fiat currency and depositing funds directly into merchants' bank accounts daily. As these services are based on Bitcoin, they can be offered for much lower fees than with PayPal or credit card networks.
	\item \textbf{Fewer risks for merchants} Bitcoin transactions are secure, irreversible, and do not contain customers’ sensitive or personal information. This protects merchants from losses caused by fraud or fraudulent chargebacks, and there is no need for PCI compliance. Merchants can easily expand to new markets where either credit cards are not available or fraud rates are unacceptably high. The net results are lower fees, larger markets, and fewer administrative costs.
	\item \textbf{Security and control} Bitcoin users are in full control of their transactions; it is impossible for merchants to force unwanted or unnoticed charges as can happen with other payment methods. Bitcoin payments can be made without personal information tied to the transaction. This offers strong protection against identity theft. Bitcoin users can also protect their money with backup and encryption.
	\item \textbf{Transparent and neutral} All information concerning the Bitcoin money supply itself is readily available on the block chain for anybody to verify and use in real-time. No individual or organization can control or manipulate the Bitcoin protocol because it is cryptographically secure. This allows the core of Bitcoin to be trusted for being completely neutral, transparent and predictable.
\end{enumerate}


Although the use of Bitcoin is advantageous in various aspects, there are also disadvantages in using it \cite{Bitcoin:2014}:
\begin{enumerate}
	\item \textbf{Degree of acceptance} Many people are still unaware of Bitcoin. Every day, more businesses accept bitcoins because they want the advantages of doing so, but the list remains small and still needs to grow in order to benefit from network effects.
	\item \textbf{Volatility} The total value of bitcoins in circulation and the number of businesses using Bitcoin are still very small compared to what they could be. Therefore, relatively small events, trades, or business activities can significantly affect the price. In theory, this volatility will decrease as Bitcoin markets and the technology matures.
	\item \textbf{Ongoing development} Bitcoin software is still in beta with many incomplete features in active development. New tools, features, and services are being developed to make Bitcoin more secure and accessible to the masses. Some of these are still not ready for everyone. Most Bitcoin businesses are new and still offer no insurance. In general, Bitcoin is still in the process of maturing.
\end{enumerate}


\subsubsection{Other cryptocurrencies}
Since the advent of Bitcoin in 2009, other cryptocurrencies followed suit. Numerous cryptocurrency specifications have been published, and a significant number of them are derived from the first implementation of Bitcoin. The two most widely adopted cryptocurrencies next to Bitcoin are Ripple and Litecoin. Table 1 shows the top 3 most widely adopted cryptocurrencies as of November 2014.


\begin{table}[h]
\centering
\begin{tabular}{ | l | l | l | l | }
\hline
  Name & Market Cap & Price / Unit & Available Supply \\ \hline \hline
  Bitcoin   & 4,970,073,323 & 365.84 & 13.5M BTC \\ \hline
  Ripple    & 415,005,339 & 0.013439 & 30.8B XRP \\ \hline
  Litecoin  & 126,835,494 & 3.67 & 34.5M LTC \\ \hline

\end{tabular}
  \caption{The top 3 most widely adopted cryptocurrencies as of November 2014. Market Cap and Price per Unit is in USD.}
\end{table}

At its core, majority of these different cryptocurrency implementations all operate in a similar fashion (technically speaking). Also, these implementations of cryptocurrencies are still at a very early stage, and are likely subject to change in the future in order to adapt to the volatile state of the market. Right now, what makes a certain cryptocurrency stand out from the rest simply boils down to one thing - adoption.

\cite{Coins.ph:06292014} discusses the bitcoin adoption in the Philippines as it is led by a well-known startup: Coins.ph. Ron Hose, co-founder and CEO of Coins.ph, firmly believes that bitcoin can help a great range of consumers in the Philippines. He explains that ``credit card penetration in the Philippines is 3\%, which means only three out of a hundred customers that land on an e-commerce site have an immediate way to pay for it'' \cite{Coins.ph:06292014}. A larger pain point of Filipinos would be transferring money from other countries. An estimate of around 21.3 billion USD was sent back to the country in 2010, a large portion of which are going to people who do not have bank accounts. These people have to collect money in person at retail locations like Western Union which requires a payment of 9\% in fees. Hose explains that ``the reason they pay this 9\% is because they don't have a bank account and likely will never have a bank account because they do not have enough savings.'' He further explains that that 9\% is equivalent in three-to-four days of income as payment just to pick up your money.

Coins.ph is a bitcoin exchange service and merchant processor which now services two major Filipino Merchants: MetroDeal and CashCashPinoy \cite{Coins.ph:03202014}. With the announcement of the new payment method accepted by MetroDeal, the Philippine's top daily deals site, Bangko Sentral ng Pilipinas (BSP) warns its consumers about the use of digital currencies such as its high volatility and lack of protection for its consumers. Furthermore, this emerging technology has yet to have regulations.

\section{Related Payment Systems}
Vision-based hand tracking systems \cite{HurstWolfgangVanWezel:2013, HurstWolfgangVriens:2013} produced encouraging results. In \cite{Hardenberg:2001}, their finger tracking process followed the following steps:

\begin{enumerate}
	\item \textbf{Image acquisition} is receiving a new image from the tracking device such as a camera.
	\item \textbf{Image differencing} is producing segmenting from the raw image the region of interest which is the hand.
	\item \textbf{Finger shape finding} is the recognition of the fingers from the segmented image. 
	\item \textbf{Hand posture classification} is the identification of the hand gesture configuration; e.g. fist, pointing gesture, pinch, etc.
	\item \textbf{Application} is the module that will test the performance of the system.
\end{enumerate}

In general, our finger tracking process follows a similar process: calibration, acquisition, segmentation, finger modeling, and the application. The hand posture classification step is removed because our system is limited to a fixed pointing gesture. This paper focuses on the segmentation and calibration processes.

There have been numerous hand tracking AR systems that have performed \textbf{segmentation} using image differencing \cite{Hardenberg:2001, Song:2008}, depth \cite{Kulshreshth:2013, Raheja:2011} or color thresholding \cite{Byron:2009, Gumpp:2006} to retrieve the region of interest: the hand. From the segmented image, systems would then perform \textbf{finger modeling} to determine the hand's position, gesture, or the position of the fingertips. Some examples of hand modeling would be finger shape detection \cite{Song:2008,Hardenberg:2001}, 3D hand modeling \cite{Byron:2009, Gumpp:2006} and pose estimation \cite{SchlattmannKahlesz:2007, Wang:2011}. The aforementioned systems were able to track the hand accurately and in real-time, but some had to use markers to aid the tracking system \cite{Chun:2013, Refinger:2007, Huynh:2009} and had bulky setups \cite{SchlattmannKahlesz:2007, Wang:2011}.


\subsection{Segmentation}

Segmentation of an image is the process of separating a certain region of interest from the rest of the image. This involves removing noise, background clutter, or shadows so that a binary image is produced. A binary image is an image made up of pixels with each having a value of either zero (0) or one (1). A general way to produce a binary image is through thresholding, which involves inspecting each pixel in an image and setting its value to 1 if the pixel falls in the threshold value pair of \( (low, high) \) and 0 otherwise. There are many different ways of performing image segmentation, but for the purpose of this study, we will be discussing smart image differencing, depth thresholding, and color segmentation.

\subsubsection{Smart Image Differencing}
Smart image differencing is an approach developed by von Hardenberg and B\'erard \cite{Hardenberg:2001} that was derived from image differencing. Image differencing is the process of comparing two images and detecting any difference or change between their pixels, and then generating a new image based on the detected differences of the pixels. The first image is referred to as the reference image, which in the context of finger tracking is usually the first frame captured by the camera. The reference image is then compared to the succeeding images or frames from the video stream being captured by the camera.
\cite{Hardenberg:2001} built upon this approach with smart image differencing. Instead of the reference image constantly being the first image captured by the camera, smart image differencing allows the reference
image to learn the background from the captured images over time. This approach shows the best results only when the background is static and the object of interest in the foreground is constantly moving. A possible drawback from this approach is that it is assumed that the constantly moving object in the foreground does not rest in one position for more than 10 seconds. Another problem in this approach is that it is difficult to detect the difference when the object of interest in the foreground moves towards non-moving dark objects in the background.

\subsubsection{Depth Thresholding}
Depth thresholding can be performed with hardware that makes use of infrared (IR) technology. Microsoft's KINECT has an IR sensor that can compute and track the depth of an image, as well as an RGB camera that can capture images in either a 1280x960 resolution at 12 frames per second (FPS), or a 640x480 resolution at
30 FPS. Using KINECT, a depth image output can also be acquired which is composed of different color groups or blobs which represent the parts of the image that have different levels of depth. Raheja, et. al. \cite{Raheja:2011} conducted a study that leveraged the KINECT to track fingertips and the center of the palm. Once the depth image was acquired, it was segmented after applying a calculated threshold value onto the color blobs of the regions where the hand points are located.

\subsubsection{Color Segmentation}
Color segmentation isolates different regions of interest in an image by differentiating them by color. Given a raw captured image, the default blue, green and red (BGR) colorspace is converted into the hue, saturation and value (HSV) colorspace, and the thresholding process is applied to the generated HSV image using a selected threshold. For this approach, there are three \( (low, high) \) pairs of threshold values, which are for the hue, saturation, and value channels. The success of the segmentation relies heavily on the selected color threshold which is necessary to differentiate the skin color of the current user. 

To determine the color threshold, a genetic algorithm was implemented to find an adequate HSV (hue-saturation-value) color threshold for color segmentation. After a number of epochs, the algorithm chooses the best color threshold within the population based on a set of heuristics or fitness score.

\subsection{Finger modeling}
Once the appropriate binary image that shows the properly segmented region of interest is produced, the next step is to detect the fingertip. The following approaches use different modeling techniques to produce a model of a hand or a finger and identify the fingertip.

\subsubsection{Von Hardenberg and B\'erard's Finger Modeling Approach}
Von Hardenberg and B\'erard\cite{Hardenberg:2001} presented an algorithm to find fingertip shapes given a binary image of a hand. Two features of a fingertip were discussed: first, the center of the fingertip is enclosed in a circle of filled (1) pixels, whose diameter is detemined by the width of the finger; and second, within a certain search square area surrounding the fingertip's inner circle, the fingertip is surrounded by a large number of unfilled (0) pixels, and is followed by a smaller number of filled pixels.

Given an (x, y) position and a region of interest, the algorithm adds the number of filled pixels around said (x, y) position within a circle of a certain diameter. The algorithm then checks for the following to determine whether or not the position is a fingertip: (1) there has to be enough filled pixels around the given position but it should not exceed a certain circle area; (2) there has to be a correct number of filled pixels vs. non-filled pixels within the search square; (3) the filled pixels outside the circle and along the search square have to be connected in a chain.

\subsubsection{$\kappa$-curvature}
The fingertip can be identified by finding the extended finger in a binary image, which is usually a protruding region in the contour of the hand. This can be determined by inspecting how much the contour curves, which can be done using the $\kappa$-curvature approach. Given a list of points in the contour, each point is inspected to see how much it curves in relation to the other points. An arbitrarily chosen $\kappa$ dictates the distance of the relative points from the selected point.

It should be noted however that the space between fingers is recognized to have high curvature as well. $\kappa$-curvature can detect pointed areas of the hand but protruding regions do not necessarily mean fingers (e.g. knuckles or other crevices in the contour). A simple heuristic to avoid these kinds of problems would be to check for the distance of the tip of the finger from the center of the closed hand.

\subsubsection{Convex Hull and Convexity Defects}
Given the contour of a hand and its different outward and inward curvatures, a convex hull is a polygon which covers the outermost points of the contour. This is akin to putting a 2D rubber band around the contour. The basic algorithm for drawing the hull involves acquiring all the external points of the contour, then for every point, getting the next two points. If the set makes a convexity defect, then the second point is ignored and the remaining points are connected. A convexity defect is the area between the convexity hull and the inward-curving part of the contour which has not been covered by the convex hull. This is usually used to determine how many fingers are raised by the hand.

% end the environment with {table*}, NOTE not {table}!

\section{System Setup}


	As seen in figure~\ref{fig:draftsetup}, the system uses mobile devices, a server such as a laptop or a computer, a router, and a projector. The mobile device is used to capture images and to perform finger tracking. This makes the setup portable and still capable of finger tracking. Moreover, using more of mobile devices the system can make a larger tracking space by tracking different physical spaces. Next is the server. It runs the game application and communicates with the clients (mobile devices), sending and receiving data to each other. Also the server is responsible for the visualization of the virtual space. Lastly, a router is used to connect the devices to one another. Its main function is connecting the clients to the server.

% \begin{figure}[!ht]
% \centering
% \includegraphics[width=2.5in]{systemsetup}
% \caption{Set up of the mobile network connected tracking system}
% \label{fig:draftsetup}
% \end{figure}

\section{Convex hull finger tracking}

%Our approach uses the convex hull that was 

From the contour of the segmented image, a convex hull is created. Dense areas of points of the convex hull are then detected using the exterior points. With these areas, a minimum spanning ellipse can be formed around the convex hull, and the two farthest points from the center of the ellipse would be considered as a finger. After that, the tracking space is limited to a smaller set so that area near the edge of image is not considered, and the last point remaining would be the fingertip. This approach deemed effective since our setup uses only a pointing hand gesture, but can be problematic if an open hand would be tracked.

\section{Evaluation}

It is important to be able to measure the performance of the different processes of finger tracking so that the system can be evaluated, providing direction for improving the system. The color segmentation and the finger tracking approach were identified to be crucial processes in finger tracking. For the scope of this paper, the evaluation methodology of the finger tracking approach will be discussed in section~\ref{finger_tracking_evaluation}.

% Do not include GA Calibration evaluation since we don't have the code + the results of that test

\subsection{Annotated dataset}
To be able to automate the performance evaluation of our finger tracking system, a dataset containing one hundred fifty (150) images of hands with the pointing gesture of ten (10) participants were collected and manually annotated. The participants were mostly Filipino and Chinese college students who had skin color ranging from dark brown to yellow. The annotation is stored in the file name, which contains the following format: \textbf{ID X x Y y.png}. Below is an example file name.

\[
1 \enskip X \enskip 307 \enskip Y \enskip 274 .png
\]


The \textit{ID} is a unique identifier of the image and \textit{x, y} are the coordinate points of the actual finger tip found in the image. See Appendix~\ref{sample_dataset} for a sample image from the dataset.

\subsection{Finger tracking}
\label{finger_tracking_evaluation}

Throughout the study, different finger tracking approaches were studied to see which approach would be feasible on the mobile device. With a \textit{FingerTracking} abstract class, different tracking approaches can be implemented and evaluated. To evaluate a selected finger tracking approach, a \textit{TrackingTester} class was developed which loads the annotated dataset containing both images and the actual point. The selected finger tracking approach is then evaluated by checking to see the distance between the \textbf{actual point} and the \textbf{detected point}. The \textit{actual point} is the annotated point on the dataset, and the \textit{detected point} is the result of the finger tracking approach used on the image. The \textit{distance} is the euclidean distance between the two points and \textit{threshold} is an arbitrary value that dictates how many pixels the detected point can vary from the actual point. The score of the tracking approach on a single image is described by the equation below:
\begin{equation}
Score = 0 < \frac{threshold}{distance} < 1
\end{equation}

Because of this design, the different tracking approaches can be evaluated rapidly, as well as different configurations of a single tracking approach. An example of different configurations could be a varying threshold values or varying rate of learning for the reference image in a smart image differencing approach. This automated evaluation aids in the selection of the best configuration for a single tracking approach.

\section{Results}



\begin{table}[h]
\centering
\caption{Performance of Convex Hull Approach}
\begin{tabular}{r|c|c|c|c|}
\cline{2-5}
\multicolumn{1}{c|}{}                    & \multicolumn{4}{c|}{\textbf{Tile size}} \\ \hline
\multicolumn{1}{|l|}{\textbf{Threshold}} & 3x3      & 4x4      & 5x5     & 6x6     \\ \hline
\multicolumn{1}{|r|}{1}                  & 66.16\%  & 77.62\%  & 84.56\% & 87.95\% \\ \hline
\multicolumn{1}{|r|}{2}                  & 89.13\%  & 93.57\%  & 94.83\% & 95.51\% \\ \hline
\multicolumn{1}{|r|}{3}                  & 94.7\%   & 95.5\%   & 95.85\% & 96.02\%   \\ \hline
\end{tabular}
\label{table:performance}
\end{table}

To test the performance of our approach, numerous tests were conducted where the threshold for the scoring was given a value of 1, 2 and 3. Furthermore, the tracking space was rescaled into a smaller resolution. A pixel in the newly scaled image would the size of a 3x3, 4x4, 5x5, or 6x6 dimension of pixels of the original image. The scores of each image would then be averaged for each test.

As seen in table~\ref{table:performance}, both the tile size and threshold affect the score of the finger tracking. Most of the tests scored at least 80.0\%, which is a promising result for the convex hull tracking approach. Moreover, out of the 150 images from the dataset, only six (6) had an internal tracking score less than 20.0\%. It should be noted that this approach may be effective for this specific setup, since only a pointed finger is being tracked.

\section{Conclusions}

The finger was successfully tracked using an approach specifically addressing the shape of a pointed finger with an accuracy of at least 80\%. There can be optimizations for the finger tracking algorithm that could further increase the performance of the system in terms of accuracy.

%\end{document}  % This is where a 'short' article might terminate

%ACKNOWLEDGMENTS are optional
\section{Acknowledgments}

We would like to thank our thesis adviser, Sir Solomon See, for guiding us throughout the course of our research.


%
% The following two commands are all you need in the
% initial runs of your .tex file to
% produce the bibliography for the citations in your paper.
\bibliographystyle{abbrv}
\bibliography{myreferences}  % sigproc.bib is the name of the Bibliography in this case
% You must have a proper ".bib" file
%  and remember to run:
% latex bibtex latex latex
% to resolve all references
%
% ACM needs 'a single self-contained file'!
%
%APPENDICES are optional
%\balancecolumns
\appendix
%Appendix A
\section{Dataset sample image}
\label{sample_dataset}
Figure~\ref{fig_sim} shows a sample image from the annotated dataset. As you can see, the captured image contains a hand with a pointing gesture, with the hand curled into a fist except the index finger which is stretched out. Notice that the captured image has a clear, uncluttered desk with a white background. Also note that there are shadows and changes in illumination.


% That's all folks!
\end{document}
